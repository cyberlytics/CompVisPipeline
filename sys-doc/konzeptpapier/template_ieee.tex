\documentclass[conference]{IEEEtran}
\IEEEoverridecommandlockouts
% The preceding line is only needed to identify funding in the first footnote. If that is unneeded, please comment it out.
\usepackage{cite}
\usepackage{amsmath,amssymb,amsfonts}
\usepackage{algorithmic}
\usepackage{graphicx}
\usepackage{textcomp}
\usepackage{xcolor}
\def\BibTeX{{\rm B\kern-.05em{\sc i\kern-.025em b}\kern-.08em
    T\kern-.1667em\lower.7ex\hbox{E}\kern-.125emX}}
\begin{document}

\title{Konzeptpapier
	\\Computer Vision Pipeline}

% Authoren	
\author{

	\IEEEauthorblockN{André Kestler}
	\IEEEauthorblockA{
		\textit{a.kestler@oth-aw.de}\\
	}
	\\
	
	\IEEEauthorblockN{Antonio Vidos}
	\IEEEauthorblockA{
		\textit{a.vidos@oth-aw.de}\\
	}
	\and
	
	\IEEEauthorblockN{Marcus Haberl}
	\IEEEauthorblockA{
		\textit{m.haberl@oth-aw.de}\\
	}
	\\
	
	\IEEEauthorblockN{Tobias Dobmeier}
	\IEEEauthorblockA{
		\textit{t.dobmeier@oth-aw.de}\\
	}
	\and
	
	\IEEEauthorblockN{Tobias Lettner}
	\IEEEauthorblockA{
		\textit{t.lettner@oth-aw.de}\\
	}
	\\
	
	\IEEEauthorblockN{Tobias Weiß}
	\IEEEauthorblockA{
		\textit{t.weiss@oth-aw.de}\\
	}
}


\maketitle

\begin{abstract}
	Ziel des Projekts ist die Implementierung einer Computer Vision Pipeline als Cloud-Anwendung. In der Benutzeroberfläche kann der Benutzer ein zu verarbeitendes Bild hochladen und eine Verarbeitungskette aus einzelnen Modulen zusammenstellen. Das Backend verarbeitet die Daten in den angegebenen Schritten und stellt dem Benutzer die Ergebnisse als verarbeitetes Bild auf der Webseite zur Verfügung.
\end{abstract}


\section{Einleitung}

This document is a model and instructions for \LaTeX.
Please observe the conference page limits.




\section{Verwandte Arbeiten}

\subsection{Subsection A}

Test A

\subsection{Subsection B}

Test B

\section{Anforderungen}

\subsection{Anforderung 1}

Anforderung 1 Beschreibung 

Akzeptanzkriterien: 
\begin{itemize}
	\item Item 1
	\item Item 2
\end{itemize}


\subsection{Anforderung 2}

Anforderung 2 Beschreibung

Akzeptanzkriterien:
\begin{itemize}
	\item Item 1
	\item Item 2
\end{itemize}





\section{Methoden}
Was wird verwendet?

\subsection{Datenfluss}
Bild hochladen  an Backend übertragen
Schritte zum verarbeiten wählen an Backend übertragen
Backend verarbeitet die Daten und sendet die Daten über ... zurück an das Frontend

\subsection{Frontend}
Frontend Technologie
Umsetzung?

\subsection{Backend}
Backend Technologie
Umsetzung?

\section{Erweiterungen}
\subsection{Erweiterung 1}
Mögliche Erweiterungen, je nach Zeit Umfang...

\subsection{Erweiterung 2}
Mögliche Erweiterungen, je nach Zeit Umfang...

\section*{Literatur}

Please number citations consecutively within brackets \cite{b1}. The 
sentence punctuation follows the bracket \cite{b2}. Refer simply to the reference 
number, as in \cite{b3}---do not use ``Ref. \cite{b3}'' or ``reference \cite{b3}'' except at 
the beginning of a sentence: ``Reference \cite{b3} was the first $\ldots$''

Number footnotes separately in superscripts. Place the actual footnote at 
the bottom of the column in which it was cited. Do not put footnotes in the 
abstract or reference list. Use letters for table footnotes.

Unless there are six authors or more give all authors' names; do not use 
``et al.''. Papers that have not been published, even if they have been 
submitted for publication, should be cited as ``unpublished'' \cite{b4}. Papers 
that have been accepted for publication should be cited as ``in press'' \cite{b5}. 
Capitalize only the first word in a paper title, except for proper nouns and 
element symbols.

For papers published in translation journals, please give the English 
citation first, followed by the original foreign-language citation \cite{b6}.

\begin{thebibliography}{00}
	\bibitem{b1} G. Eason, B. Noble, and I. N. Sneddon, ``On certain integrals of Lipschitz-Hankel type involving products of Bessel functions,'' Phil. Trans. Roy. Soc. London, vol. A247, pp. 529--551, April 1955.
	\bibitem{b2} J. Clerk Maxwell, A Treatise on Electricity and Magnetism, 3rd ed., vol. 2. Oxford: Clarendon, 1892, pp.68--73.
	\bibitem{b3} I. S. Jacobs and C. P. Bean, ``Fine particles, thin films and exchange anisotropy,'' in Magnetism, vol. III, G. T. Rado and H. Suhl, Eds. New York: Academic, 1963, pp. 271--350.
	\bibitem{b4} K. Elissa, ``Title of paper if known,'' unpublished.
	\bibitem{b5} R. Nicole, ``Title of paper with only first word capitalized,'' J. Name Stand. Abbrev., in press.
	\bibitem{b6} Y. Yorozu, M. Hirano, K. Oka, and Y. Tagawa, ``Electron spectroscopy studies on magneto-optical media and plastic substrate interface,'' IEEE Transl. J. Magn. Japan, vol. 2, pp. 740--741, August 1987 [Digests 9th Annual Conf. Magnetics Japan, p. 301, 1982].
	\bibitem{b7} M. Young, The Technical Writer's Handbook. Mill Valley, CA: University Science, 1989.
\end{thebibliography}
\vspace{12pt}

\end{document}
