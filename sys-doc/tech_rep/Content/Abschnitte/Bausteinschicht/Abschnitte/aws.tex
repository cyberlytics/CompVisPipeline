\subsection{AWS: Amazon Web Services}
Im Rahmen des Projekts wurde Amazon Web Sevices (AWS) verwendet, um die Anwendung mithilfe einer EC2-Instanz zu verteilen und die Daten mithilfe eines S3-Buckets zentral zu speichern. Auf den
S3-Bucket wird mithilfe, der im Frontend und Backend erstellten Klasse \glqq S3Manager\grqq{} zugegriffen. Der S3-Bucket fungiert als Datenschnittstelle zwischen dem Frontend und dem Backend.

Die Zugriffsrechte wurden in der AWS-Konsole über den Identity und Access Management (IAM) konfiguriert. Dabei wurden feingranulare Rechte an die Benutzer 
vergeben. Dies soll sicherstellen, dass nur auf den autorisierte Benutzer auf den \glqq bdcc-fatcat-data\grqq{} Bucket zugreifen können. Diese Sicherheitsmaßnahme soll die anderen Bereiche der Cloud schützen, falls ein Bereich kompromittiert wird.

Durch die Verwendung von AWS und den beiden S3Manager Klassen konnte eine skalierbare Lösung zur Bildspeicherung und -verwaltug bereitgestellt werden.