\subsection{AWS: Amazon Web Services}
Im Rahmen des Projekts wurde Amazon Web Sevices (AWS) verwendet, um die Anwendung mithilfe einer EC2-Instanz zu hosten und die Daten mithilfe eines S3-Buckets zentral zu speichern. Auf den
S3-Bucket wird mithilfe, der im Frontend und Backend erstellten Klasse \textit{S3Manager} zugegriffen. Die Klasse basiert auf den Funktionen der Bibliothek boto3 \cite{boto3} und der aws-sdk \cite{awsSdk}. Der S3-Bucket dient als gemeinsamer Speicher zum Austausch von Bildern zwischen Frontend und Backend. 

Die Zugriffsrechte wurden in der AWS-Konsole über den Identity und Access Management (IAM) konfiguriert. Dabei wurden feingranulare Rechte an die Benutzer 
vergeben. Dies soll sicherstellen, dass nur autorisierte Benutzer auf den \textit{bdcc-fatcat-data} Bucket zugreifen können. Diese Sicherheitsmaßnahme soll die anderen Bereiche der Cloud schützen, falls ein Bereich kompromittiert wird.

Durch die Verwendung von AWS und den beiden S3Manager Klassen konnte eine skalierbare Lösung zur Bildspeicherung und -verwaltung bereitgestellt werden.