\subsection{Frontend: Benutzerschnittstelle}
Das in React geschriebene Frontend besteht aus einer Hauptkomponente, welche aus insgesamt sieben Kindkomponenten besteht.
Ein \textit{Header}, welcher die Möglichkeit bietet die Seite zu refreshen, Informationen über die Internetseite einzuholen und den Darkmode zu aktivieren, befindet sich am oberen Bildschirmrand.
Für berechtigte Personen gibt es ein User-Icon, welches eine berechtigte Person zum administrativen Bereich führt.
Eine weitere Komponente im oberen Bereich ermöglicht dem User den Upload eines eigenen oder eines Beispielbildes.
Zwei weitere Komponenten am linken Bildschirmrand, \textit{Selected Image} und \textit{Imagedetails}, sind dafür zuständig das hochgeladene Bild anzuzeigen, sowie die Bildinformationen in Form eines Histogrammes anzuzeigen.
Die Komponente \textit{Availible steps} am rechten Bildschirmrand stellt dem Benutzer alle möglichen Bildoperationen zur Verfügung.
Möchte dieser eine Bildoperation anwenden, muss eine Karte aus der Komponente per Drag and Drop in die dafür vorgesehene Komponente \textit{Pipeline} im Zentrum abgelegt werden.
Möchte der Benutzer die Pipeline starten, kann dieser das durch das Klicken der Schaltfläche \textit{Start Pipeline} erreichen.
Das Frontend schickt nun alle benötigten Informationen an das Backend.
Das Klicken des Symbols \textit{i} liefert dem Benutzer Informationen zu der jeweiligen Bildoperation.
Hat der Benutzer eine Pipelinekarte abgelegt, kann der Benutzer über das Klicken des nach unten gerichteten Pfeils die Parameter der Bildoperation konfigurieren.
Klickt der Benutzer auf das Aug-Icon, wird das bearbeitete Bild bis zur ausgewählten Operation angezeigt.
Dies ermöglicht es dem Benutzer, Schritt für Schritt Ergebnisse aus den einzelnen Operationen nachvollziehen zu können.