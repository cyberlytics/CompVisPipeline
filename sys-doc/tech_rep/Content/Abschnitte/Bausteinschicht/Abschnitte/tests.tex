\subsection{Testabdeckung}
Im \textbf{Frontend} wurden Unit-Tests mit Hilfe des Testframeworks \glqq{}jest\grqq{}\cite{jest} implementiert. Durch die Verwendung der \glqq{}--coverage\grqq{}-Einstellung wurde die Testabdeckung ermittelt, um sicherzustellen, dass eine möglichst umfassende Prüfung des Codes erfolgt.

Im \textbf{Backend} wurden ebenfalls Unit-Tests eingesetzt, diesmal jedoch mit dem Testframework \glqq{}pytest\grqq{}\cite{pytest}. Auch hier wurde die Testabdeckung mit der \glqq{}--cov\grqq{}-Einstellung überwacht, um sicherzustellen, dass der Testumfang den größtmöglichen Teil des Backend-Codes abdeckt.

Die Durchführung von Unit-Tests und die Überprüfung der Testabdeckung tragen wesentlich zur Verbesserung der Softwarequalität bei. Sie ermöglichen das frühzeitige Erkennen von potenziellen Fehlern oder Problemen. Darüber hinaus bieten sie eine zusätzliche Sicherheitsebene, um Codeänderungen ohne das Risiko unerwünschter Nebeneffekte durchführen zu können. Die systematische Durchführung von Tests gewährleistet somit eine höhere Stabilität und Robustheit der Cloud-Anwendung.
