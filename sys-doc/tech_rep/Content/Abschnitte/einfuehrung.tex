\section{Einführung und Ziele}
Die digitale Bildverarbeitung spielt eine entscheidende Rolle bei der automatischen Extraktion von Informationen aus Bildern. Dabei werden verschiedene Schritte wie Filterung und Segmentierung durchgeführt, um die gewünschten Ergebnisse zu erzielen. Diese Schritte müssen für verschiedene Anwendungsfälle getestet und die Parameter angepasst werden. Um diesen Prozess zu vereinfachen, wurde die \glqq Computer Vision Pipeline\grqq{} als Cloud-Anwendung entwickelt. Sie ermöglicht es dem Benutzer, Bilder hochzuladen und mit Hilfe einer intuitiven Benutzeroberfläche eine Verarbeitungskette zu erstellen.

Der vorliegende technical Report beschreibt die Architektur der entwickelten \glqq Computer Vision Pipeline\grqq{}. Dabei werden die allgemeine Architektur, die Bausteinsicht und die Verteilungssicht erläutert. Des Weiteren werden die verwendeten Entwicklungswerkzeuge beschrieben. Abschließend wird ein Fazit gezogen und ein Ausblick gegeben.