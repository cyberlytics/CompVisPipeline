\section{Architektur Allgemein}
Für unsere Anwendung haben wir eine Architektur gewählt, die auf einer Client-Server-Kommunikation basiert. Dabei haben wir klare Frontend- und Backend-Komponenten definiert, die jeweils in separaten Containern entwickelt werden. Die Kommunikation zwischen Frontend und Backend erfolgt über eine REST-Schnittstelle. Diese Schnittstelle ermöglicht einen effizienten Informationsaustausch und eine standardisierte Kommunikation zwischen den beiden Komponenten. Unsere Applikation wird in der AWS Cloud gehostet. Hier nutzen wir einen S3 Bucket zur Speicherung der (Zwischen-)Ergebnisse und eine EC2 Instanz für das Hosting. Der S3 Bucket bietet uns eine zuverlässige Lösung für die Speicherung der Ergebnisse, während die EC2 Instanz das Hosting unserer Anwendung ermöglicht. Mit dieser Architektur bieten wir Anwendern eine benutzerfreundliche Möglichkeit, eine Computer Vision Pipeline aufzubauen und Bilder entsprechend zu verarbeiten. Die Verwendung von Client-Server Kommunikation ermöglicht eine effektive Zusammenarbeit zwischen Frontend und Backend. Die Containerisierung erleichtert das Management der Entwicklungsumgebung. Die Bereitstellung in der AWS Cloud mit einem S3 Bucket und einer EC2 Instanz garantiert eine skalierbare und zuverlässige Ausführung unserer Anwendung.