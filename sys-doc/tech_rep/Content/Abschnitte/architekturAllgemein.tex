\section{Architektur Allgemein}
Die Anwendung basiert auf einer Client-Server-Architektur, bei welcher die Frontend- und Backend-Komponenten in separaten Docker-Containern entwickelt werden. Die Kommunikation zwischen Frontend und Backend erfolgt über eine REST-Schnittstelle, die einen effizienten Informationsaustausch und eine standardisierte Kommunikation ermöglicht. Im Frontend hat der Benutzer die Möglichkeit eine Pipeline aus verschiedenen Bearbeitungsschritten (Steps) zusammenzustellen, die dann im Backend auf ein vom Benutzer hochgeladenes Bild angewendet werden. Der Benutzer hat außerdem die Möglichkeit, das Standard-Bild oder ein zufälliges Bild zu laden, welches von einer künstlichen Intelligenz generiert wurde. Die Generierung des KI-Bildes erfolgt über externe API-Schnittstellen. Die Anwendung wird in der AWS Cloud gehostet und nutzt einen S3 Bucket zur zuverlässigen Speicherung der (Zwischen-)Ergebnisse. Die Hosting-Funktionalität wird durch eine EC2-Instanz bereitgestellt.