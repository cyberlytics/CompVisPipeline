\section{Fazit und Ausblick}
Das Projekt Computer Vision Pipeline bietet den Benutzern eine einfache Anwendung zur Verwendung von gängigen Bildverarbeitungsschritten.

Abschließend lässt sich festhalten, dass das Projektteam trotz der unterschiedlichen Vorkenntnisse erfolgreich zusammenarbeitete. Die Einarbeitung in das Frontend-Framework React kostete zwar Zeit, aber führte letztendlich zu einer ansprechenden Benutzeroberfläche. Auch die Implementierung von Tests erwies sich als herausfordernd. Zur Qualitätssicherung des Systems sind diese aber von großer Bedeutung.

Ein Ausblick auf das Projekt zeigt vielversprechende Möglichkeiten zur Erweiterung der Anwendung. Durch die angelegte Pipelinestruktur im Backend, können mit wenig Aufwand weitere Funktionen eingebaut werden. Da personenbezogene Daten, wie die Bilder, verarbeitet werden, sollte in Betracht gezogen werden die Anwendung um eine Authentifizierung für den Benutzer zu erweitern. Dadurch sollte sichergestellt werden, dass nur der Benutzer auf seine Bilder Zugriff hat. Außerdem sollte die Nutzung von Session-Tokens für AWS im Frontend implementiert werden. Dadurch wird es ermöglicht die AWS-Credentials (die Zugangsdaten) nur temporär an das Frontend zu übertragen. Dies wurde aufgrund der Zeit nicht in den Produktiv-Code eingefügt. 

Die Kommunikation zwischen allen Komponenten erfolgt in der aktuellen Implementierung über das unverschlüsselte \textit{HTTP}-Protokoll. Um eine sichere Datenübertragung zu gewährleisten sollte das verschlüsselte \textit{HTTPS}-Protokoll verwendet werden. Neben der bereits beschriebenen unverschlüsselten Kommunikation zwischen Frontend und Backend, stellt auch der direkte Zugriff auf den S3-Bucket durch das Frontend in der aktuellen Implementierung ein Sicherheitsrisiko dar.