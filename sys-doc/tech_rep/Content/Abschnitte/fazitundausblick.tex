\section{Fazit und Ausblick}
Das Projekt Computer Vision Pipeline soll den Benutzern eine einfache Schnittstelle zur Verwendung von gängigen Bildverarbeitungsbibliotheken bieten.

Im Fazit lässt sich festhalten, dass das Projektteam trotz der unterschiedlichen Vorkenntnisse erfolgreich zusammengearbeitet hat. Die Einarbeitung in das Frontend-Framework React hat zwar Zeit gekostet, aber letztendlich zu einer ansprechenden und leistungsstarken Benutzeroberfläche geführt. Auch die Implementierung von Tests erwies sich als herausfordernd. Zur Qualitätssicherung des Sytems sind diese aber von großer Bedeutung.

Ein Ausblick auf das Projekt zeigt vieversprechende Möglichkeiten zur Erweiterung der Anwendung. Durch die angelegte Pipelinestruktur im Backend, können mit wenig Aufwand weitere Funktionen eingebaut werden. Da personenbezogene Daten, wie die Bilder, verarbeitet werden, sollte in Betracht gezogen werden die Anwendung um einen Authentifizierung für den Benutzer zu erweitern. Dadurch sollte sichergestellt werden, dass nur der Benutzer auf seine Bilder zugriff hat. Außerdem sollten die Session-Tokens für AWS im Frontend implementiert werden, dass die AWS-Credentials (die Zugangsdaten) temporär an das Frontend übertragen werden. Dies wurde aufgrund der Zeit nicht mehr zu Ende implementiert.