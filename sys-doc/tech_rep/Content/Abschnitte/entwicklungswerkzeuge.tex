\section{entwicklungswerkzeuge}
Für die Entwicklung des Projekts wurde verschiedene Werkzeuge verwendet, um ein möglichst reibungsloses und effizientes Arbeiten zu gewährleisten.

\subsection{Paketverwaltung}
Die Paketverwaltung spielt eine eintscheidende Rolle bei der Verwaltung der Abhängigkeiten und dem Versionsmanagement der einzelnen Bibliotheken.

Im Backend wurde \glqq{}pip\grqq{} verwendet. Dies ist die Standardpaketverwaltung für Python. \glqq{}Pip\grqq{} ermöglicht die einfache Installation und Aktualisierung von Python-Paketen. Im Frontend erfolgte die Paketverwaltung mithilfe von  \glqq{}npm\grqq{} (Node Package Manager). Dadurch konnten die benötigten Pakete für das Frontend effizient verwaltet werden.

\subsection{Frontend}
Das Frontend des Projekts basiert auf React \cite{react}. Dies ist eine leistungsstarke JavaScript-Bibliothek, die zur Entwicklung von Benutzeroberflächen dient. React ermöglichte uns, eine modulare Gestaltung der Oberfläche, was zu einer besseren Wartbarkeit des Codes führte. Die einzelnen React-Komponenten wurden in JSX (einer Erweiterung für JavaScript) geschrieben. 

\subsection{Backend}
Das Backend des Projekts basiert auf Python und dem Flask-Framework. Python hat uns mit seiner breiten Palette an Paketen die Entwicklung der Bildverarbeitung erleichtert. Flask zeichnet sich durch seine einfache Handhabung und Flexibilität aus REST-APIs zu implementieren und die erforderlichen Routen im Backend festzulegen.

\subsection{Container}
Um eine effiziente Bereitstellung des Systems zu gewährleisten, wurden Container in Form von Docker-Compose \cite{docker} implementiert. Dabei existiert ein Container für das Frontend und ein Container für das Backend. Durch die Verwendung von Containern wurde sichergestellt, dass alle erforderlichen Abhängigkeiten, Bibliotheken und Konfigurationen zentral festegelegt sind und Plattform unabgängig ausgeführt werden können.